\documentclass[10pt,spanish]{article}

\usepackage[spanish]{babel}
\usepackage[utf8]{inputenc}
\usepackage{amsmath, amsthm}
\usepackage{amsfonts, amssymb, latexsym}
\usepackage{enumerate}
\usepackage[usenames, dvipsnames]{color}
\usepackage{colortbl}
\usepackage{minted}
\usepackage[left=3cm, right=3cm]{geometry}

\usepackage[bookmarks=true,
            bookmarksnumbered=false, % true means bookmarks in
                                     % left window are numbered
            bookmarksopen=false,     % true means only level 1
                                     % are displayed.
            colorlinks=true,
            linkcolor=webblue]{hyperref}
\definecolor{webgreen}{rgb}{0, 0.5, 0} % less intense green
\definecolor{webblue}{rgb}{0, 0, 0.5}  % less intense blue
\definecolor{webred}{rgb}{0.5, 0, 0}   % less intense red

\setlength{\parindent}{0pt}
\setlength{\parskip}{1ex plus 0.5ex minus 0.2ex}

%%%%% Para cambiar el tipo de letra en el título de la sección %%%%%%%%%%%
\usepackage{sectsty}
\sectionfont{\fontfamily{pag}\selectfont}
\subsectionfont{\fontfamily{pag}\selectfont}
\subsubsectionfont{\fontfamily{pag}\selectfont}

\definecolor{rojo}{rgb}{0.8, 0.0, 0.0}
\definecolor{azul}{rgb}{0.2, 0.2, 0.6}
\definecolor{temacuatro}{rgb}{0.0, 0.62, 0.38}
\definecolor{temacinco}{rgb}{0.93, 0.35, 0.0}
\definecolor{temaseis}{rgb}{0.6, 0.4, 0.8}
\definecolor{temasiete}{rgb}{0.0, 0.47, 0.75}

% \usepackage[default]{frcursive}
% \usepackage[T1]{fontenc}

%Definimos autor y título
\title{\fontfamily{pag}\selectfont \bfseries \Huge \color{rojo} Aritmética Modular}
\author{\fontfamily{pag}\selectfont \bfseries \LARGE Marta Gómez}

\begin{document}
\maketitle

\renewcommand{\tablename}{Tabla}

\section{\textcolor{rojo}Ejercicio 1}
\textit{Implementa el \textbf{\textcolor{rojo}{Algoritmo extendido de Euclides}} para el cálculo del máximo común divisor: dados dos enteros $a$ y $b$, encuentra $u, v \in \mathbb{Z}$ tales que $au + bv$ sea el máximo común divisor de $a$ y $b$.}

Para resolver este ejercicio, he desarrollado dos funciones en \textit{\textcolor{rojo}{Haskell}}:

\begin{enumerate}[---]
    \item \texttt{extended\_euclides}: esta función sirve como interfaz para el usuario y comprueba el caso base, que $b=0$. En el caso de que no se de el caso base, se pasa a utilizar la siguiente función.

\begin{minted}[linenos, frame=lines, label={extended\_euclides}]{haskell}
extended_euclides :: Integral a => a -> a -> (a, a, a)
extended_euclides a 0 = (a, 1, 0)
extended_euclides a b = extended_euclides_tabla a b 1 0 0 1

\end{minted}

    \item \texttt{extended\_euclides\_tabla}: esta  función implementa la tabla para el cálculo del algoritmo extendido de Euclides. A diferencia de la anterior, necesita como entrada los valores de $(x2, x1, y2, y1)$ además de $a$ y $b$. 

\begin{minted}[linenos, frame=lines, label={extended\_euclides\_tabla}]{haskell}
extended_euclides_tabla :: Integral a => a -> a -> a -> a -> a -> a -> (a, a, a)
extended_euclides_tabla a 0 x2 _ y2 _ = (a, x2, y2)
extended_euclides_tabla a b x2 x1 y2 y1 = extended_euclides_tabla b r x1 x y1 y
            where
                q = a `div` b
                r = a - q*b
                x = x2 - q*x1
                y = y2 - q*y1

\end{minted}
\end{enumerate}

\section{\textcolor{rojo}Ejercicio 2}
\textit{Usando el cálculo del ejercicio anterior, escribe una función que calcule $a^{-1}\; mod \; b$ para cualesquiera $a$ y $b$ que sean primos relativos.}

Para resolver este ejercicio he calculado el $u$ tal que $au + bv$. Esto es posible debido a que $a$ y $b$ son primos relativos y, por tanto, $mcd(a,b) = 1$. Si no se cumple esta condición, el programa devuelve $-1$.

Una vez obtenido $u$, he calculado su valor módulo $b$.

\begin{minted}[linenos, frame=lines, label={inverse}]{haskell}
inverse :: Integral a => a -> a -> a
inverse a b
    | r == 1 = i `mod` b
    | otherwise = -1
        where
            (r,i,_) = extended_euclides a b
\end{minted}

\section{\textcolor{rojo}Ejercicio 3}
\textit{Escribe una función que calcule $a^b \; mod \; n$ para cualesquiera $a, b \in \mathbb{Z}$ que sean primos relativos. La implementación debería tener en cuenta la representación binaria de $b$.}

Para hacer este ejercicio, podríamos primero realizar la potencia $a^b$ y después, calcular el módulo $n$ del resultado. Ahora bien, para un $b$ muy grande esta aproximación es muy costosa.

Es por eso que debe considerarse la representación binaria de $b = \sum_{i=0}^t b_i 2^i$. Usando dicha represetación binaria, $a^b$ puede calcularse como:



\end{document}
