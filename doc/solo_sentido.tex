\documentclass[10pt,spanish]{article}

\usepackage[spanish]{babel}
\usepackage[utf8]{inputenc}
\usepackage{amsmath, amsthm}
\usepackage{amsfonts, amssymb, latexsym}
\usepackage{enumerate}
\usepackage[usenames, dvipsnames]{color}
\usepackage{colortbl}
\usepackage{minted}
\usepackage[left=3cm, right=3cm]{geometry}
\usepackage{cancel}
\usepackage{graphicx}
\usepackage{subfigure}

\usepackage[bookmarks=true,
            bookmarksnumbered=false, % true means bookmarks in
                                     % left window are numbered
            bookmarksopen=false,     % true means only level 1
                                     % are displayed.
            colorlinks=true,
            linkcolor=webblue]{hyperref}
\definecolor{webgreen}{rgb}{0, 0.5, 0} % less intense green
\definecolor{webblue}{rgb}{0, 0, 0.5}  % less intense blue
\definecolor{webred}{rgb}{0.5, 0, 0}   % less intense red

\setlength{\parindent}{0pt}
\setlength{\parskip}{1ex plus 0.5ex minus 0.2ex}

%%%%% Para cambiar el tipo de letra en el título de la sección %%%%%%%%%%%
\usepackage{sectsty}
\sectionfont{\fontfamily{pag}\selectfont}
\subsectionfont{\fontfamily{pag}\selectfont}
\subsubsectionfont{\fontfamily{pag}\selectfont}

\definecolor{azul}{rgb}{0.0, 0.44, 1.0}

% \usepackage[default]{frcursive}
% \usepackage[T1]{fontenc}

%Definimos autor y título
\title{\fontfamily{pag}\selectfont \bfseries \Huge \color{azul} Funciones de un sólo sentido}
\author{\fontfamily{pag}\selectfont \bfseries \LARGE Marta Gómez}

\begin{document}
\maketitle
\tableofcontents

\renewcommand{\tablename}{Tabla}

\section{\textcolor{azul}Ejercicio 1}
\textit{Sea $(a_1,\cdots,a_k)$ una secuencia \textbf{\textcolor{azul}{super-creciente}} de números positivos (la suma de todos los términos que preceden a $a_i$ es menor que $a_i$ para todo $i$). Elige un $n > \sum a_i$ y $u$ un entero positivo tal que $gcd(n,u) = 1$. Define $a_i^* = ua_i \pmod n$. La función \textbf{\textcolor{azul}{mochila}} (\textbf{\textcolor{azul}{knapsack}}) asociada a $(a_1^*,\cdots,a_k^*)$ es}

\begin{displaymath}
    f: \mathbb{Z}_2^k \rightarrow \mathbb{N}, f(x_1,\cdots,x_k) = \sum_{i=1}^k x_i a_i^*
\end{displaymath}

\textit{Implementa esta función y su inversa. La llave púlica es $(a_1^*,\cdots,a_k^*)$, mientras que la privada (y la puerta de atrás es) $((a_1,\cdots,a_k),n,u)$.}

\subsection{\textcolor{azul}Generación de una secuencia super-creciente aleatoria}
Para generar una secuencia \textit{\textcolor{azul}{super-creciente}} aleatoria, debemos asegurarnos de que dicha secuencia cumpla con la siguiente condición:

\begin{displaymath}
    b_i > \sum_{j=1}^{i-1} b_i \qquad\ \forall i, 2 \leq i \leq n
\end{displaymath}

Mi implementación en Haskell toma un primer elemento de forma aleatoria en el intervalo $[2,20]$ y, construye el resto de elementos a partir del primero, multiplicando el último elemento de la lista por dos. Como parámetro, recibe el número de elementos que tendrá la secuencia.

\begin{minted}[frame=single, label={genera\_secuencia}]{haskell}
genera_secuencia :: (Integral a, Random a) => a -> [a]
genera_secuencia t = take (fromIntegral t) $ iterate (\x -> x*2) r
    where
        r = fst $ randomR (2,20) $ mkStdGen (238012)
\end{minted}

Un ejemplo de secuencia obtenida es la siguiente, que empieza por 12:

\begin{displaymath}
    [12,24,48,96,192,384,768,1536,3072,6144]
\end{displaymath}

\subsection{\textcolor{azul}Generación de claves}
Para generar claves me he basado en el capítulo 4.5 del libro \textit{\textcolor{azul}{Notes on Cryptography}}. El código Haskell desarrollado es:

\begin{minted}[frame=single, label={Generación de claves}]{haskell}
is_prime_relative :: (Integral a) => a -> a -> Bool
is_prime_relative a b = x == 1
    where
        (x,_,_) = extended_euclides a b

mochi_gen_claves :: (Integral a, Random a) => [a] -> ([a], a, a, [a])
mochi_gen_claves s = (a,n,u,s)
    where
        n  = (sum s) * 2
        u  = head $ dropWhile (\x -> not (is_prime_relative x n)) $ randomRs (1,n-1) 
             $ mkStdGen (28165137)
        a  = map (\x -> x*u `mod` n) s
\end{minted}

En primer lugar, calculo $n = 2 \cdot \sum s_i$. En lugar de calcular un número aleatorio mayor a la sumatoria, calculo el doble de ésta porque en Haskell los aleatorios son algo complicados. 

Una vez he calculado $n$, paso a calcular $u$ generando números aleatorios hasta obtener uno que sea primo relativo con $n$.

Por último, calculo $a^*$ de la siguiente forma:

\begin{displaymath}
    a_i^* = ua_i \pmod n \qquad\ \forall i=1,\ldots,k 
\end{displaymath}

Devuelvo una tupla de cuatro elementos: el primero es la clave pública, $a^*$, y los tres siguientes forman la clave privada, $n, u, s$.

\subsection{\textcolor{azul}Cifrado}
Para cifrar un mensaje necesitamos la clave pública, $a^*$, generada previamente y el mensaje a cifrar.

\end{document}